\maketitle
\section{Different Measures of Centrality}

\paragraph{Dependency Centrality}\mbox{} \medskip \\
The question of what node in a network is the most central one is often interesting
and important for making the optimal decision for many cases in multiple fields. 
The four most common centrality measures are degree-, closeness-, betweenness- and eigenvector 
centrality. The following describes those measures and will introduce an additional kind of 
centrality named dependency centrality. Networks are represented as graphs.

\paragraph{Degree Centrality}\mbox{} \medskip \\
The most central nodes are the ones that are connected to the most other nodes.
Which more precisely means the nodes for which the amount of edges that contain them is the highest. 
An application of this measure would be to ask what twitter account has the most followers.
Here twitter accounts are the nodes and the fact that one account follows another corresponds 
to an edge between those accounts/nodes.

\paragraph{Closeness Centrality}\mbox{} \medskip \\
The most central nodes are the ones that are on average the closest to every other node
in the graph with closeness meaning the minimum number of edges one has to travel to get to another node.

\paragraph{Betweenness Centrality}\mbox{} \medskip \\
For this measure to find the most central node one first considers every shortest
path between every node. The node that lies on the highest amount of such paths is the
most central one. An example where this measure would be interesting is in a network
of computers if an attacker wants to sniff as much traffic between those computers aspossible.
The most ideal target would be the most central one according to betweenness centrality.

\paragraph{Eigenvector Centrality}\mbox{} \medskip \\
The most central node is recursively defined as the node that is connected to the highest number
of other nodes that themselves have a high degree of centrality again. In a network of facebook
accounts, with the fact that one account is befriended with another as an edge, eigenvector centrality
would be a good measure to determine what account can influence the most users. For this problem you
do not want to use degree centrality since the account that has the most friends does not necessarily
have the most reach. The account with the most influence is rather the one that has a lot of friends
that themselves have a lot of friends and so on.

\paragraph{Dependency Centrality}\mbox{} \medskip \\


\section{Calculating Centrality}

\paragraph{Eigenvector Centrality Calculation}\mbox{} \medskip \\
Finding the most central node according to eigenvector centrality is less straight forward than 
according to degree-, closeness- or betweenness centrality. To do so one can represent the graph 
as an adjacency matrix of size $n \times n$ with $n$ as the number of nodes. Each node of the graph 
corresponds to a row and a column of the matrix. If there is an edge from node $a$ to node $b$
in the graph, then in the matrix there is a one in row $a$ and column $b$.

\begin {center}
\begin {tikzpicture}[-latex ,auto ,node distance =2 cm and 2cm ,on grid ,
semithick ,
state/.style ={ circle ,top color =white, draw, minimum width =.5 cm}]
\node[state] (C) {$1$};
\node[state] (A) [above left=of C] {$0$};
\node[state] (B) [above right =of C] {$2$};
\path (C) edge [] (A);
\path (A) edge [] (C);
\path (A) edge [] (B);
\path (B) edge [] (A);
\path (C) edge [] (B);
\path (B) edge [] (C);
\end{tikzpicture}
\end{center}


For this adjacency matrix one can calculate the eigenvectors $x$ and the corresponding 
eigenvalues $ \lambda $ that solve the equation $Ax = \lambda x$. The important eigenvector is the one 
with the highest eigenvalue. In this vector the index of the highest entry is the 
index of the row and the column that corresponds to the most central node.

graph und adjazenzmatrix als visualisierung

\paragraph{Dependency Centrality Calculation}\mbox{} \medskip \\